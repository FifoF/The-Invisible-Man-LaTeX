\label{ch:19}
\SetRectoHeadText{\addfontfeature{Letters=ResetAll,LetterSpace=0,Scale=0.9}\itshape Certain First Principles}
\begin{ChapterStart}
\vspace*{2\nbs}

\ChapterTitle{CHAPTER\, NINETEEN}
\vspace{1.5\nbs}
\ChapterSubtitle{Certain First}
\vspace{0.75\nbs}
\ChapterSubtitle{Principles}
\end{ChapterStart}

\dropcap[lines=2,ante=“]{W}\kern-4pt\textsc{hat’s the matter}?”\ asked Kemp, when the Invisible Man admitted him.

“Nothing,” was the answer.

“But, confound it! The smash?”

“Fit of temper,” said the Invisible Man. “Forgot this arm; and it’s sore.”

“You’re rather liable to that sort of thing.”

“I am.”

Kemp walked across the room and picked up the fragments of broken glass. “All the facts are out about you,” said Kemp, standing up with the glass in his hand; “all that happened in Iping, and down the hill. The world has become aware of its invisible citizen. But no one knows you are here.”

The Invisible Man swore.

“The secret’s out. I gather it was a secret. I don’t know what your plans are, but of course I’m anxious to help you.”

The Invisible Man sat down on the bed.

“There’s breakfast upstairs,” said Kemp, speaking as easily as possible, and he was delighted to find his strange guest rose willingly. Kemp led the way up the narrow staircase to the belvedere.

“Before we can do anything else,” said Kemp, “I must understand a little more about this invisibility of yours.” He had sat down, after one nervous glance out of the window, with the air of a man who has talking to do. His doubts of the sanity of the entire business flashed and vanished again as he looked across to where Griffin sat at the breakfast-table,—a headless, handless dressing-gown, wiping unseen lips on a miraculously held serviette.

“It’s simple enough—and credible enough,” said Griffin, putting the serviette aside and leaning the invisible head on an invisible hand.

“No doubt, to you, but—” Kemp laughed.

“Well, yes; to me it seemed wonderful at first, no doubt. But now, great God!— But we will do great things yet! I came on the stuff first at Chesilstowe.”

“Chesilstowe?”

“I went there after I left London. You know I dropped medicine and took up physics? \emph{No!}—well, I did. \emph{Light}— fascinated me.”

“Ah!”

“Optical density! The whole subject is a network of riddles—a network with solutions glimmering elusively through. And being but two and twenty and full of enthusiasm, I said, ‘I will devote my life to this. This is worth while.’ You know what fools we are at two and twenty?”

“Fools then or fools now,” said Kemp.

“As though Knowing could be any satisfaction to a man!

“But I went to work—like a nigger. And I had hardly worked and thought about the matter six months before light came through one of the meshes suddenly—blindingly! I found a general principle of pigments and refraction,—a formula, a geometrical expression involving four dimensions. Fools, common men, even common mathematicians, do not know anything of what some general expression may mean to the student of molecular physics. In the books—the books that Tramp has hidden—there are marvels, miracles! But this was not a method, it was an idea, that might lead to a method by which it would be possible, without changing any other property of matter,—except, in some instances, colours,—to lower the refractive index of a substance, solid or liquid, to that of air—so far as all practical purposes are concerned.”

“Phew!”\ said Kemp. “That’s odd! But still I don’t see quite— I can understand that thereby you could spoil a valuable stone, but personal invisibility is a far cry.”

“Precisely,” said Griffin. “But consider: Visibility depends on the action of the visible bodies on light. Either a body absorbs light, or it reflects or refracts it, or does all these things. If it neither reflects nor refracts nor absorbs light, it cannot of itself be visible. You see an opaque red box, for instance, because the colour absorbs some of the light and reflects the rest, all the red part of the light, to you. If it did not absorb any particular part of the light, but reflected it all, then it would be a shining white box. Silver! A diamond box would neither absorb much of the light nor reflect much from the general surface, but just here and there where the surfaces were favourable the light would be reflected and refracted, so that you would get a brilliant appearance of flashing reflections and translucencies,—a sort of skeleton of light. A glass box would not be so brilliant, not so clearly visible, as a diamond box, because there would be less refraction and reflection. See that? From certain points of view you would see quite clearly through it. Some kinds of glass would be more visible than others, a box of flint glass would be brighter than a box of ordinary window glass. A box of very thin common glass would be hard to see in a bad light, because it would absorb hardly any light and refract and reflect very little. And if you put a sheet of common white glass in water, still more if you put it in some denser liquid than water, it would vanish almost altogether, because light passing from water to glass is only slightly refracted or reflected or indeed affected in any way. It is almost as invisible as a jet of coal gas or hydrogen is in air. And for precisely the same reason!”

“Yes,” said Kemp, “that is pretty plain sailing.”

“And here is another fact you will know to be true. If a sheet of glass is smashed, Kemp, and beaten into a powder, it becomes much more visible while it is in the air; it becomes at last an opaque white powder. This is because the powdering multiplies the surfaces of the glass at which refraction and reflection occur. In the sheet of glass there are only two surfaces; in the powder the light is reflected or refracted by each grain it passes through, and very little gets right through the powder. But if the white powdered glass is put into water, it forthwith vanishes. The powdered glass and water have much the same refractive index; that is, the light undergoes very little refraction or reflection in passing from one to the other.

“You make the glass invisible by putting it into a liquid of nearly the same refractive index; a transparent thing becomes invisible if it is put in any medium of almost the same refractive index. And if you will consider only a second, you will see also that the powder of glass might be made to vanish in air, if its refractive index could be made the same as that of air; for then there would be no refraction or reflection as the light passed from glass to air.”

“Yes, yes,” said Kemp. “But a man’s not powdered glass!”

“No,” said Griffin. “\emph{He’s more transparent!}”

“Nonsense!”

“That from a doctor! How one forgets! Have you already forgotten your physics, in ten years? Just think of all the things that are transparent and seem not to be so. Paper, for instance, is made up of transparent fibres, and it is white and opaque only for the same reason that a powder of glass is white and opaque. Oil white paper, fill up the interstices between the particles with oil so that there is no longer refraction or reflection except at the surfaces, and it becomes as transparent as glass. And not only paper, but cotton fibre, linen fibre, wool fibre, woody fibre, and \emph{bone}, Kemp, \emph{flesh}, Kemp, \emph{hair}, Kemp, \emph{nails} and \emph{nerves}, Kemp, in fact the whole fabric of a man except the red of his blood and the black pigment of hair, are all made up of transparent, colourless tissue. So little suffices to make us visible one to the other. For the most part the fibres of a living creature are no more opaque than water.”

“Great Heavens!”\ cried Kemp. “Of course, of course! I was thinking only last night of the sea larvæ and all jelly-fish!”

“\emph{Now} you have me! And all that I knew and had in mind a year after I left London—six years ago. But I kept it to myself I had to do my work under frightful disadvantages. Oliver, my professor, was a scientific bounder, a journalist by instinct, a thief of ideas,—he was always prying! And you know the knavish system of the scientific world. I simply would not publish, and let him share my credit I went on working, I got nearer and nearer making my formula into an experiment, a reality. I told no living soul, because I meant to flash my work upon the world with crushing effect,—to become famous at a blow. I took up the question of pigments to fill up certain gaps. And suddenly, not by design but by accident, I made a discovery in physiology.”

“Yes?”

“You know the red colouring matter of blood; it can be made white—colourless—and remain with all the functions it has now!”

Kemp gave a cry of incredulous amazement.

The Invisible Man rose and began pacing the little study. “You may well exclaim. I remember that night. It was late at night,—in the daytime one was bothered with the gaping, silly students,—and I worked then sometimes till dawn. It came suddenly, splendid and complete into my mind. I was alone; the laboratory was still, with the tall lights burning brightly and silently. In all my great moments I have been alone. ‘One could make an animal—a tissue—transparent! One could make it invisible! All except the pigments—I could be invisible!’\ I said, suddenly realizing what it meant to be an albino with such knowledge. It was overwhelming. I left the filtering I was doing, and went and stared out of the great window at the stars. ‘I could be invisible!’\ I repeated.

“To do such a thing would be to transcend magic. And I beheld, unclouded by doubt, a magnificent vision of all that invisibility might mean to a man,—the mystery, the power, the freedom. Drawbacks I saw none. You have only to think! And I, a shabby, poverty-struck, hemmed-in demonstrator, teaching fools in a provincial college, might suddenly become—this. I ask you, Kemp, if you— Anyone, I tell you, would have flung himself upon that research. And I worked three years, and every mountain of difficulty I toiled over showed another from its summit. The infinite details! And the exasperation,—a professor, a provincial professor, always prying. ‘When are you going to publish this work of yours?’\ was his everlasting question. And the students, the cramped means! Three years I had of it—

“And after three years of secrecy and exasperation, I found that to complete it was impossible,—impossible.”

“How?”\ asked Kemp.

“Money,” said the Invisible Man, and went again to stare out of the window.

He turned round abruptly. “I robbed the old man—robbed my father.

“The money was not his, and he shot himself.”