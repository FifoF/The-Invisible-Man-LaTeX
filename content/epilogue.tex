\cleartorecto
\label{ch:epilogue}
\SetRectoHeadText{\addfontfeature{Letters=ResetAll,LetterSpace=0,Scale=0.9}The Epilogue}
\begin{ChapterStart}
\vspace*{2\nbs}

\ChapterTitle{}
\vspace{1.5\nbs}
\ChapterSubtitle{The Epilogue}
\end{ChapterStart}

\dropcap[lines=2]{S}\kern-3pt\textsc{o ends the story} of the strange and evil experiment of the Invisible Man. And if you would learn more of him you must go to a little inn near Port Stowe and talk to the landlord. The sign of the inn is an empty board save for a hat and boots, and the name is the title of this story. The landlord is a short and corpulent little man with a nose of cylindrical protrusion, wiry hair, and a sporadic rosiness of visage. Drink generously, and he will tell you generously of all the things that happened to him after that time, and of how the lawyers tried to do him out of the treasure found upon him.\looseness=-1

“When they found they couldn’t prove who’s money was which, I’m blessed,” he says, “if they didn’t try to make me out a blooming treasure trove! Do I \emph{look} like a Treasure Trove? And then a gentleman gave me a guinea a night to tell the story at the Empire Music ’all—just tell ’em in my own words—barring one.”

And if you want to cut off the flow of his reminiscences abruptly, you can always do so by asking if there weren’t three manuscript books in the story. He admits there were and proceeds to explain, with asseverations that everybody thinks \emph{he} has ’em! But bless you!\ he hasn’t. “The Invisible Man it was took ’em off to hide ’em when I cut and ran for Port Stowe. It’s that Mr.\ Kemp put people on with the idea of \emph{my} having ’em.”

And then he subsides into a pensive state, watches you furtively, bustles nervously with glasses, and presently leaves the bar.

He is a bachelor man—his tastes were ever bachelor, and there are no women folk in the house. Outwardly he buttons—it is expected of him—but in his more vital privacies, in the matter of braces for example, he still turns to string. He conducts his house without enterprise, but with eminent decorum. His movements are slow, and he is a great thinker. But he has a reputation for wisdom and for a respectable parsimony in the village, and his knowledge of the roads of the South of England would beat Cobbett.

And on Sunday mornings, every Sunday morning, all the year round, while he is closed to the outer world, and every night after ten, he goes into his bar parlour, bearing a glass of gin faintly tinged with water, and having placed this down, he locks the door and examines the blinds, and even looks under the table. And then, being satisfied of his solitude, he unlocks the cupboard and a box in the cupboard and a drawer in that box, and produces three volumes bound in brown leather, and places them solemnly in the middle of the table. The covers are weather-worn and tinged with an algal green—for once they sojourned in a ditch and some of the pages have been washed blank by dirty water. The landlord sits down in an armchair, fills a long clay pipe slowly—gloating over the books the while. Then he pulls one towards him and opens it, and begins to study it—turning over the leaves backwards and forwards.

His brows are knit and his lips move painfully. “Hex, little two up in the air, cross and a fiddle-de-dee. Lord!\ what a one he was for intellect!”

Presently he relaxes and leans back, and blinks through his smoke across the room at things invisible to other eyes. “Full of secrets,” he says. “Wonderful secrets!

“Once I get the haul of them— \emph{Lord!}

“I wouldn’t do what \emph{he} did; I’d just—well!” He pulls at his pipe.

So he lapses into a dream, the undying wonderful dream of his life. And though Kemp has fished unceasingly, and Adye has questioned closely, no human being save the landlord knows those books are there, with the subtle secret of invisibility and a dozen other strange secrets written therein. And none other will know of them until he dies.