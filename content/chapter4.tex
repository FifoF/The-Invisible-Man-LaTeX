\label{ch:04}
\SetRectoHeadText{\addfontfeature{Letters=ResetAll,LetterSpace=0,Scale=0.9}\itshape Mr. \kern-2ptCuss Interviews the Stranger}
\begin{ChapterStart}
\vspace*{2\nbs}

\ChapterTitle{CHAPTER\, FOUR}
\vspace{1.5\nbs}
\ChapterSubtitle{Mr. Cuss Interviews}
\vspace{0.75\nbs}
\ChapterSubtitle{the Stranger}
\end{ChapterStart}

\dropcap[lines=2]{I}\textsc{have told} the circumstances of the stranger’s arrival in Iping with a certain fulness of detail, in order that the curious impression he created may be understood by the reader. But excepting two odd incidents, the circumstances of his stay until the extraordinary day of the Club Festival may be passed over very cursorily. There were a number of skirmishes with Mrs.\ Hall on matters of domestic discipline, but in every case until late in April, when the first signs of penury began, he over-rode her by the easy expedient of an extra payment. Hall did not like him, and whenever he dared he talked of the advisability of getting rid of him; but he showed his dislike chiefly by concealing it ostentatiously, and avoiding his visitor as much as possible. “Wait till the summer,” said Mrs.\ Hall, sagely, “when the artisks are beginning to come. Then we’ll see. He may be a bit overbearing, but bills settled punctual is bills settled punctual, whatever you like to say.”

The stranger did not go to church, and indeed made no difference between Sunday and the irreligious days, even in costume. He worked, as Mrs.\ Hall thought, very fitfully. Some days he would come down early and be continuously busy. On others he would rise late, pace his room, fretting audibly for hours together, smoke, sleep in the armchair by the fire. Communication with the world beyond the village he had none. His temper continued very uncertain; for the most part his manner was that of a man suffering under almost unendurable provocation, and once or twice things were snapped, torn, crushed, or broken in spasmodic gusts of violence. He seemed under a chronic irritation of the greatest intensity. His habit of talking to himself in a low voice grew steadily upon him, but though Mrs.\ Hall listened conscientiously she could make neither head nor tail of what she heard.

He rarely went abroad by daylight, but at twilight he would go out muffled up invisibly, whether the weather were cold or not, and he chose the loneliest paths and those most overshadowed by trees and banks. His goggling spectacles and ghastly bandaged face under the penthouse of his hat, came with a disagreeable suddenness out of the darkness upon one or two home-going labourers, and Teddy Henfrey, tumbling out of the Scarlet Coat one night, at half-past nine, was scared shamefully by the stranger’s skull-like head (he was walking hat in hand) lit by the sudden light of the opened inn door. Such children as saw him at nightfall dreamt of bogies, and it seemed doubtful whether he disliked boys more than they disliked him, or the reverse,—but there was certainly a vivid dislike enough on either side.

It was inevitable that a person of so remarkable an appearance and bearing should form a frequent topic in such a village as Iping. Opinion was greatly divided about his occupation. Mrs.\ Hall was sensitive on the point. When questioned, she explained very carefully that he was an “experimental investigator,” going gingerly over the syllables as one who dreads pitfalls. When asked what an experimental investigator was, she would say with a touch of superiority that most educated people knew such things as that, and would thus explain that he “discovered things.” Her visitor had had an accident, she said, which temporarily discoloured his face and hands, and being of a sensitive disposition, he was averse to any public notice of the fact.

Out of her hearing there was a view largely entertained that he was a criminal trying to escape from justice by wrapping himself up so as to conceal himself altogether from the eye of the police. This idea sprang from the brain of Mr.\ Teddy Henfrey. No crime of any magnitude dating from the middle or end of February was known to have occurred. Elaborated in the imagination of Mr.\ Gould, the probationary assistant in the National School, this theory took the form that the stranger was an Anarchist in disguise, preparing explosives, and he resolved to undertake such detective operations as his time permitted. These consisted for the most part in looking very hard at the stranger whenever they met, or in asking people who had never seen the stranger, leading questions about him. But he detected nothing.

Another school of opinion followed Mr.\ Fearenside, and either accepted the piebald view or some modification of it; as, for instance, Silas Durgan, who was heard to assert that “if he choses to show enself at fairs he’d make his fortune in no time,” and being a bit of a theologian, compared the stranger to the man with the one talent. Yet another view explained the entire matter by regarding the stranger as a harmless lunatic. That had the advantage of accounting for everything straight away.

Between these main groups there were waverers and compromisers. Sussex folk have few superstitions, and it was only after the events of early April that the thought of the supernatural was first whispered in the village. Even then it was only credited among the women folks.

But whatever they thought of him, people in Iping, on the whole, agreed in disliking him. His irritability, though it might have been comprehensible to an urban brain-worker, was an amazing thing to these quiet Sussex villagers. The frantic gesticulations they surprised now and then, the headlong pace after nightfall that swept him upon them round quiet corners, the inhuman bludgeoning of all the tentative advances of curiosity, the taste for twilight that led to the closing of doors, the pulling down of blinds, the extinction of candles and lamps,—who could agree with such goings on? They drew aside as he passed down the village, and when he had gone by, young humourists would up with coat-collars and down with hat-brims, and go pacing nervously after him in imitation of his occult bearing. There was a song popular at that time called the “Bogey Man”; Miss Statchell sang it at the schoolroom concert (in aid of the church lamps), and thereafter whenever one or two of the villagers were gathered together and the stranger appeared, a bar or so of this tune, more or less sharp or flat, was whistled in the midst of them. Also belated little children would call “Bogey Man!”\ after him, and make off tremulously elated.

Cuss, the general practitioner, was devoured by curiosity. The bandages excited his professional interest, the report of the thousand and one bottles aroused his jealous regard. All through April and May he coveted an opportunity of talking to the stranger, and at last, towards Whitsuntide, he could stand it no longer, but hit upon the subscription-list for a village nurse as an excuse. He was surprised to find that Mr.\ Hall did not know his guest’s name. “He give a name,” said Mrs.\ Hall,—an assertion which was quite unfounded,—“but I didn’t rightly hear it.” She thought it seemed so silly not to know the man’s name.

Cuss rapped at the parlour door and entered. There was a fairly audible imprecation from within. “Pardon my intrusion,” said Cuss, and then the door closed and cut Mrs.\ Hall off from the rest of the conversation.

She could hear the murmur of voices for the next ten minutes, then a cry of surprise, a stirring of feet, a chair flung aside, a bark of laughter, quick steps to the door, and Cuss appeared, his face white, his eyes staring over his shoulder. He left the door open behind him, and without looking at her strode across the hall and went down the steps, and she heard his feet hurrying along the road. He carried his hat in his hand. She stood behind the door, looking at the open door of the parlour. Then she heard the stranger laughing quietly, and then his footsteps came across, the room. She could not see his face where she stood. The parlour door slammed, and the place was silent again.

Cuss went straight up the village to Bunting the vicar. “Am I mad?”\ Cuss began abruptly, as he entered the shabby little study. “Do I look like an insane person?”

“What’s happened?”\ said the vicar, putting the ammonite on the loose sheets of his forthcoming sermon.

“That chap at the inn—”

“Well?”

“Give me something to drink,” said Cuss, and he sat down.

When his nerves had been steadied by a glass of cheap sherry,—the only drink the good vicar had available,—he told him of the interview he had just had. “Went in,” he gasped, “and began to demand a subscription for that Nurse Fund. He’d stuck his hands in his pockets as I came in, and he sat down lumpily in his chair. Sniffed. I told him I’d heard he took an interest in scientific things. He said yes. Sniffed again. Kept on sniffing all the time; evidently recently caught an infernal cold. No wonder, wrapped up like that! I developed the nurse idea, and all the while kept my eyes open. Bottles—chemicals—everywhere. Balance, test-tubes in stands, and a smell of—evening primrose. Would he subscribe? Said he’d consider it. Asked him, point-blank, was he researching. Said he was. A long research? Got quite cross. ‘A damnable long research,’ said he, blowing the cork out, so to speak. ‘Oh,’ said I. And out came the grievance. The man was just on the boil, and my question boiled him over. He had been given a prescription, most valuable prescription—what for he wouldn’t say. Was it medical? ‘Damn you! What are you fishing after?’ I apologised. Dignified sniff and cough. He resumed. He’d read it. Five ingredients. Put it down; turned his head. Draught of air from window lifted the paper. Swish, rustle. He was working in a room with an open fireplace, he said. Saw a flicker, and there was the prescription burning and lifting chimneyward. Rushed towards it just as it whisked up chimney. So! Just at that point, to illustrate his story, out came his arm.”

“Well?”

“No hand, just an empty sleeve. Lord! I thought, \emph{that’s} a deformity! Got a cork arm, I suppose, and has taken it off. Then, I thought, there’s something odd in that. What the devil keeps that sleeve up and open, if there’s nothing in it? There was nothing in it, I tell you. Nothing down it, right down to the joint. I could see right down it to the elbow, and there was a glimmer of light shining through a tear of the cloth. ‘Good God!’\ I said. Then he stopped. Stared at me with those black goggles of his, and then at his sleeve.”

“Well?”

“That’s all. He never said a word; just glared, and put his sleeve back in his pocket quickly. ‘I was saying,’ said he, ‘that there was the prescription burning, wasn’t I?’ Interrogative cough. ‘How the devil,’ said I, ‘can you move an empty sleeve like that?’ ‘Empty sleeve?’ ‘Yes,’ said I, ‘an empty sleeve.’

“\kern1pt‘It’s an empty sleeve, is it? You saw it was an empty sleeve?’ He stood up right away. I stood up too. He came towards me in three very slow steps, and stood quite close. Sniffed venomously. I didn’t flinch, though I’m hanged if that bandaged knob of his, and those blinkers, aren’t enough to unnerve any one, coming quietly up to you.

“\kern1pt‘You said it was an empty sleeve?’\ he said. ‘Certainly,’ I said. At staring and saying nothing a barefaced man, unspectacled, starts scratch. Then very quietly he pulled his sleeve out of his pocket again, and raised his arm towards me as though he would show it to me again. He did it very, very slowly. I looked at it. Seemed an age. ‘Well?’\ said I, clearing my throat, ‘there’s nothing in it.’ Had to say something. I was beginning to feel frightened. I could see right down it. He extended it straight towards me, slowly, slowly, just like that, until the cuff was six inches from my face. Queer thing to see an empty sleeve come at you like that! And then—”

“Well?”

“Something—exactly like a finger and thumb it felt—\linebreak{}nipped my nose.”

Bunting began to laugh.

“There wasn’t anything there!”\ said Cuss, his voice running up into a shriek at the “there.” “It’s all very well for you to laugh, but I tell you I was so startled, I hit his cuff hard, and turned round, and cut out of the room—I left him—”

Cuss stopped. There was no mistaking the sincerity of his panic. He turned round in a helpless way and took a second glass of the excellent vicar’s very inferior sherry. “When I hit his cuff,” said Cuss, “I tell you, it felt exactly like hitting an arm. And there wasn’t an arm! There wasn’t the ghost of an arm!”

Mr.\ Bunting thought it over. He looked suspiciously at Cuss. “It’s a most remarkable story,” he said. He looked very wise and grave indeed. “It’s really,” said Mr.\ Bunting with judicial emphasis, “a most remarkable story.”