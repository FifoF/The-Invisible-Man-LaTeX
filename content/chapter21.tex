\label{ch:21}
\SetRectoHeadText{\addfontfeature{Letters=ResetAll,LetterSpace=0,Scale=0.9}\itshape In Oxford Street}
\begin{ChapterStart}
\vspace*{2\nbs}

\ChapterTitle{CHAPTER\, TWENTY-ONE}
\vspace{1.5\nbs}
\ChapterSubtitle{In Oxford Street}
\end{ChapterStart}

\dropcap[lines=2]{I}\kern-4pt\textsc{n going downstairs} the first time I found an unexpected difficulty because I could not see my feet; indeed I stumbled twice, and there was an unaccustomed clumsiness in gripping the bolt. By not looking down, however, I managed to walk on the level passably well.

“My mood, I say, was one of exaltation. I felt as a seeing man might do, with padded feet and noiseless clothes, in a city of the blind. I experienced a wild impulse to jest, to startle people, to clap men on the back, fling people’s hats astray, and generally revel in my extraordinary advantage.

“But hardly had I emerged upon Great Portland Street, however (my lodging was close to the big draper’s shop there), when I heard a clashing concussion and was hit violently behind, and turning saw a man carrying a basket of soda-water syphons, and looking in amazement at his burden. Although the blow had really hurt me, I found something so irresistible in his astonishment that I laughed aloud. ‘The devil’s in the basket,’ I said, and suddenly twisted it out of his hand. He let go incontinently, and I swung the whole weight into the air.

“But a fool of a cabman, standing outside a public house, made a sudden rush for this, and his extending fingers took me with excruciating violence under the ear. I let the whole down with a smash on the cabman, and then, with shouts and the clatter of feet about me, people coming out of shops, vehicles pulling up, I realised what I had done for myself, and cursing my folly, backed against a shop window and prepared to dodge out of the confusion. In a moment I should be wedged into a crowd and inevitably discovered. I pushed by a butcher boy, who luckily did not turn to see the nothingness that shoved him aside, and dodged behind the cabman’s four-wheeler. I do not know how they settled the business. I hurried straight across the road, which was happily clear, and hardly heeding which way I went, in the fright of detection the incident had given me, plunged into the afternoon throng of Oxford Street

“I tried to get into the stream of people, but they were too thick for me, and in a moment my heels were being trodden upon. I took to the gutter, the roughness of which I found painful to my feet, and forthwith the shaft of a crawling hansom dug me forcibly under the shoulder blade, reminding me that I was already bruised severely. I staggered out of the way of the cab, avoided a perambulator by a convulsive movement, and found myself behind the hansom. A happy thought saved me, and as this drove slowly along I followed in its immediate wake, trembling and astonished at the turn of my adventure. And not only trembling, but shivering. It was a bright day in January and I was stark naked and the thin slime of mud that covered the road was freezing. Foolish as it seems to me now, I had not reckoned that, transparent or not, I was still amenable to the weather and all its consequences.

“Then suddenly a bright idea came into my head. I ran round and got into the cab. And so, shivering, scared, and sniffing with the first intimations of a cold, and with the bruises in the small of my back growing upon my attention, I drove slowly along Oxford Street and past Tottenham Court Road. My mood was as different from that in which I had sallied forth ten minutes ago as it is possible to imagine. \emph{This} invisibility indeed! The one thought that possessed me was—how was I to get out of the scrape I was in.

“We crawled past Mudie’s, and there a tall woman with five or six yellow-labelled books hailed my cab, and I sprang out just in time to escape her, shaving a railway van narrowly in my flight. I made off up the roadway to Bloomsbury Square, intending to strike north past the Museum and so get into the quiet district. I was now cruelly chilled, and the strangeness of my situation so unnerved me that I whimpered as I ran. At the northward corner of the Square a little white dog ran out of the Pharmaceutical Society’s offices, and incontinently made for me, nose down.

“I had never realised it before, but the nose is to the mind of a dog what the eye is to the mind of a seeing man. Dogs perceive the scent of a man moving as men perceive his vision. This brute began barking and leaping, showing, as it seemed to me, only too plainly that he was aware of me. I crossed Great Russell Street, glancing over my shoulder as I did so, and went some way along Montagu Street before I realised what I was running towards.

“Then I became aware of a blare of music, and looking along the street saw a number of people advancing out of Russell Square, red shirts, and the banner of the Salvation Army to the fore. Such a crowd, chanting in the roadway and scoffing on the pavement, I could not hope to penetrate, and dreading to go back and farther from home again, and deciding on the spur of the moment, I ran up the white steps of a house facing the museum railings, and stood there until the crowd should have passed. Happily the dog stopped at the noise of the band too, hesitated, and turned tail, running back to Bloomsbury Square again.

“On came the band, bawling with unconscious irony some hymn about ‘When shall we see his Face?’ and it seemed an interminable time to me before the tide of the crowd washed along the pavement by me. Thud, thud, thud, came the drum with a vibrating resonance, and for the moment I did not notice two urchins stopping at the railings by me. ‘See ’em,’ said one. ‘See what?’\ said the other. ‘Why—them footmarks—\emph{bare}. Like what you makes in mud.’

“I looked down and saw the youngsters had stopped and were gaping at the muddy footmarks I had left behind me up the newly whitened steps. The passing people elbowed and jostled them, but their confounded intelligence was arrested. ‘Thud, thud, thud. When, thud, shall we see, thud, his face, thud, thud.’ ‘There’s a barefoot man gone up them steps, or I don’t know nothing,’ said one. ‘And he ain’t never come down again. And his foot was a-bleeding.’

“The thick of the crowd had already passed. ‘Looky there, Ted,’ quoth the younger of the detectives, with the sharpness of surprise in his voice, and pointed straight to my feet. I looked down and saw at once the dim suggestion of their outline sketched in splashes of mud. For a moment I was paralysed.

“\kern1pt‘Why, that’s rum,’ said the elder. ‘Dashed rum! It’s just like the ghost of a foot, ain’t it?’ He hesitated and advanced with outstretched hand. A man pulled up short to see what he was catching, and then a girl. In another moment he would have touched me. Then I saw what to do. I made a step, the boy started back with an exclamation, and with a rapid movement I swung myself over into the portico of the next house. But the smaller boy was sharp-eyed enough to follow the movement, and before I was well down the steps and upon the pavement, he had recovered from his momentary astonishment and was shouting out that the feet had gone over the wall.

“They rushed round and saw my new footmarks flash into being on the lower step and upon the pavement. ‘What’s up?’\ asked someone. ‘Feet! Look! Feet running!’ Everybody in the road, except my three pursuers, was pouring along after the Salvation Army, and this blow not only impeded me but them. There was an eddy of surprise and interrogation. At the cost of bowling over one young fellow I got through, and in another moment I was rushing headlong round the circuit of Russell Square, with six or seven astonished people following my footmarks. There was no time for explanation, or else the whole host would have been after me.

“Twice I doubled round corners, thrice I crossed the road and came back on my tracks, and then, as my feet grew hot and dry, the damp impressions began to fade. At last I had a breathing space and rubbed my feet clean with my hands, and so got away altogether. The last I saw of the chase was a little group of a dozen people perhaps, studying with infinite perplexity a slowly drying footprint that had resulted from a puddle in Tavistock Square,—a footprint as isolated and incomprehensible to them as Crusoe’s solitary discovery.

“This running warmed me to a certain extent, and I went on with a better courage through the maze of less frequented roads that runs hereabouts. My back had now become very stiff and sore, my tonsils were painful from the cabman’s fingers, and the skin of my neck had been scratched by his nails; my feet hurt exceedingly and I was lame from a little cut on one foot. I saw in time a blind man approaching me, and fled limping, for I feared his subtle intuitions. Once or twice accidental collisions occurred and I left people amazed, with unaccountable curses ringing in their ears. Then came something silent and quiet against my face, and across the Square fell a thin veil of slowly falling flakes of snow. I had caught a cold, and do as I would I could not avoid an occasional sneeze. And every dog that came in sight, with its pointing nose and curious sniffing, was a terror to me.

“Then came men and boys running, first one and then others, and shouting as they ran. It was a fire. They ran in the direction of my lodging, and looking back down a street I saw a mass of black smoke streaming up above the roofs and telephone wires. It was my lodging burning; my clothes, my apparatus, all my resources indeed, except my cheque-book and the three volumes of memoranda that awaited me in Great Portland Street, were there. burning! I had burnt my boats—if ever a man did! The place was blazing.”

The Invisible Man paused and thought Kemp glanced nervously out of the window. “Yes?”\ he said. “Go on.”